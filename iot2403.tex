%%
%% 研究報告用スイッチ
%% [techrep]
%%
%% 欧文表記無しのスイッチ(etitle,eabstractは任意)
%% [noauthor]
%%

%\documentclass[submit,techrep]{ipsj}
%\documentclass[submit,techrep,noauthor]{ipsj}
\documentclass[submit,techrep]{ipsj}
\usepackage[T1]{fontenc}
\usepackage{lmodern}
\usepackage{textcomp}
\usepackage{latexsym}
\usepackage{listings}
\usepackage{url}
\usepackage{xcolor}
\def\newblock{\ }%

\renewcommand{\floatpagefraction}{.8}

\definecolor{codegreen}{rgb}{0,0.6,0}
\definecolor{codegray}{rgb}{0.5,0.5,0.5}
\definecolor{codepurple}{rgb}{0.58,0,0.82}
\definecolor{backcolour}{rgb}{0.95,0.95,0.92}

\lstdefinestyle{mystyle}{
    backgroundcolor=\color{backcolour},   
    commentstyle=\color{codegreen},
    keywordstyle=\color{magenta},
    numberstyle=\tiny\color{codegray},
    stringstyle=\color{codepurple},
    basicstyle=\ttfamily\footnotesize,
    breakatwhitespace=false,         
    breaklines=true,                 
    captionpos=b,                    
    keepspaces=true,                 
    numbers=left,                    
    numbersep=5pt,                  
    showspaces=false,                
    showstringspaces=false,
    showtabs=false,                  
    tabsize=2
}
\lstset{style=mystyle}


\usepackage[dvipdfmx]{graphicx}
\usepackage{latexsym}

\def\Underline{\setbox0\hbox\bgroup\let\\\endUnderline}
\def\endUnderline{\vphantom{y}\egroup\smash{\underline{\box0}}\\}
\def\|{\verb|}
%

\newcommand{\reffig}[1]{図\ref{#1}}
\newcommand{\reftab}[1]{表\ref{#1}}


\newcommand{\epsfig}[4]{
\begin{figure}[tb]
  \begin{center}
    \includegraphics[#2]{#1}
  \end{center}
  \caption{#3}
  \label{#4}
\end{figure}}

\newcommand{\epsfigfull}[4]{
\begin{figure*}[t]
  \begin{center}
    \includegraphics[#2]{#1}
  \end{center}
  \caption{#3}
  \label{#4}
\end{figure*}}

%\setcounter{巻数}{59}%vol59=2018
%\setcounter{号数}{10}
%\setcounter{page}{1}

\newcommand{\kbs}{Kubernetes}

\begin{document}

\title{パブリックコンテナサービスを用いた\\超分散テストベッドの構築}

\etitle{A Prototype Implementation of Computing Continuum Testbed \\
using Public Cloud Container Service}

\affiliate{aist}{産業技術総合研究所\\
National Institute of Advanced Industrial Science and Technology}

\affiliate{utsukuba}{筑波大学\\University of Tsukuba}

\author{董 允治}{Yunzhi Dong}{utsukuba,aist}[]
\author{中田 秀基}{Hidemoto Nakada}{aist,utsukuba}[hide-nakada@aist.go.jp]
\author{谷村 勇輔}{Yusuke Tanimura}{aist,utsukuba}[yusuke.tanimura@aist.go.jp]


\begin{abstract}
  IoTセンサの普及に伴いセンサデータの爆発的増大が想定される。このような環境では
  エッジにおいて前処理を行うことでデータ量を低減するとともにクラウドでの処理を軽
  減するアプローチが有効であると考えられる。このような環境で動作するミドルウェア
  の負荷に対する特性を評価するには大規模なテストベッドが必要だが、実機でこのよう
  なテストベッドを用意するのはさまざまな観点から現実的ではない。
  シミュレータを使用する方法も考えられるが、各モジュールへの負荷を検証することは
  できない。
  我々はクラウド上のコンテナサービスを利用することで、テストベッドを
  構築する方法を提案する。オーケストレーションサービスを用いることで
  容易に短時間で大規模なテスト環境を構築できることを確認した。
\end{abstract}

\begin{jkeyword}
  Kubernetes, パブリッククラウド, 大規模テストベッド
\end{jkeyword}

\begin{eabstract}
  With the proliferation of IoT sensors,
  an explosive increase in sensor data is expected. 
  In such an environment, an approach that reduces the amount of 
  data by pre-processing at the edge and reduces processing 
  in the cloud is considered effective.
  A large-scale testbed is necessary to evaluate the load characteristics 
  of middleware running in such an environment, 
  but preparing such a testbed on actual equipment is not 
  realistic from various perspectives.
  We propose a method to build a testbed by using container services in the public cloud.
  We have confirmed that a large-scale test environment 
  can be easily built in a short time by using an orchestration service.
\end{eabstract}

\begin{ekeyword}
  Kuberanetes, public cloud, large-scale testbed
\end{ekeyword}

\maketitle

%1
\section{はじめに}

IoTセンサの普及に伴いセンサデータの爆発的増大が起こりつつある。
このような状況においては、センサからクラウドへ直接情報を送信すると
クラウドに過負荷がかかることが予想される。
これに対してセンサとクラウドの中間にエッジと呼ばれる層を追加し、
エッジとクラウドで適切に負荷を分散することで、クラウドに負荷が集中
するのを抑制しようという試みが提案されている\cite{iot}\cite{Edge-computing}。
我々は、このような試みの一貫として、センサからのデータをエッジで
集約することでクラウドへの負荷を低減するシステムを提案し、
その実現性を検討してきた\cite{tou-os}。

しかしこのようなシステムの大規模な環境での評価は容易ではない。
多数のノードから構成される実験環境を確立することは
それ自体技術的にも経済的にも困難である。
Grid5000\cite{grid5000}のような例はあるが、維持管理のコストは
膨大で、持続可能ではない。
%
SimGrid\cite{simgrid}などのシミュレータを利用する方法も考えられるが、
多くのシミュレータはネットワークのみに着目しており、個々のノード上で
動作するモジュールの過負荷を評価することはできない。

これに対して、我々はクラウド上のコンテナオーケストレーション
サービスを利用することで、
テストベッドを構築する方法を提案した\cite{tou-hpc}\cite{tou-imcom}。
本稿では、このアプローチをさらに進め、テストベッドの構築と破棄を自動化し、
Jupyter Notebook\cite{jupyternotebook}から制御する方法を提案する。
提案システムでは、パブリッククラウド上へのテスト環境を構築し、
実験を実行し、テスト環境を破棄する作業を手元のPC環境から
容易に行う事ができる。

本稿の構成は以下のとおりである。
\ref{sec:background}節では、本稿で用いる\kbs やAmazon EKS、
テスト対象となるMQTTに関して説明する。
\ref{sec:testbed}節では、提案システム上で構成するテストベッドについて
概説する。
\ref{sec:proposal}節で、提案システムの構成について説明し、
%\ref{sec:evaluation}節で評価を行う。
\ref{sec:conclusion}節で本稿をまとめ、将来の課題を述べる。

\section{背景}\label{sec:background}



\subsection{\kbs}
\subsubsection{\kbs の概要}
\kbs\cite{k8s}は、複数のコンテナを管理するコンテナオーケストレータの一つで、
デファクトスタンダードとして広く用いられている。
コンテナオーケストレータは、
個別の機能を持つ多数の小さいサービスを組み合わせることで
一つのアプリケーションを構成するマイクロサービスが普及とともに一般化した。
コンテナとはアプリケーションもしくはサービスを、
その依存する環境とともにパッケージしたものだ。
コンテナオーケストレータは、
このようなコンテナの集合を一体として管理監視し、起動終了を行う。
たとえば、故障したコンテナがあれば、同じ機能を持つ別のコンテナを
自動的に起動するセルフヒーリング機能を持つ。

\kbs はコントロールプレーンとワーカーノードから構成される。
コントロールプレーンには、ユーザからの入力を受け付けるAPIサーバや
状態を管理するデータベースに相当するetcd、
クラスタの情報を収集するコントロールマネージャ、
コンテナを実行するノードを決定するスケジューラが存在する。

\epsfig{figs/kubernetes.pdf}{width=8.5cm}{\kbs の概要}{kubernetes}

個々のワーカノードには複数のポッドを動作させる事ができる。
ポッドは個別のIPアドレスを持つ単位で、
ポッドの中にさらに複数のコンテナを持つ事ができる。
この様子を\reffig{kubernetes}に示す。

\kbs はさまざまなコンテナエンジンをサポートするが、
本稿ではDockerを用いる。

\subsubsection{クライアントツール}
\kbs ユーザはコントロールプレーンのAPIサーバに接続する
何らかのクライアントツールを用いて、\kbs クラスタを制御する。
コマンドラインツールの\verb|kubectl|がよく用いられるが、
APIライブラリやUIツールも存在する。


%\subsubsection{Deployment}
%\kbs では1つのサービスのレプリカを複数個用意し、自動的に起
%これをDeploymentと呼ぶ


\subsubsection{ConfigMapとSecret}

例えばアクセス先のIPアドレスや、認証情報をコンテナイメージに組み込むと、
これらが変更されるたびにコンテナイメージのリビルドが必要になり非効率である。
\kbs では、ConfigMapとSecretと呼ばれる機能を用いることで、
コンテナに対して起動時に外部から設定情報を与えることができる。


\subsubsection{DaemonSet}
DaemonSetを用いると、すべて(もしくは一部)のノードで自動的に実行する
ポッドを指定することができる。
例えばログ収集やノードを監視するデーモンなどはすべてのノードで動作することが望ましい。
このようなデーモンを実行するポッドを自動起動するものがDaemonSetである。

\subsection{Amazon EKS}
\kbs はオンプレミス環境でも広く使用されているが、小規模な
組織で運用するのはそれほど容易ではない。
これに対してパブリッククラウド上で\kbs をサポートするサービスが
登場している。
%
Amazon EKS(Elastic Kubernetes Service)\cite{EKS}はその一つで、
KubernetesをAmazonクラウド上で実行するサービスである。
同様に、Google Computing Services にはGoogle Kubernetes Engine(GKE)\cite{gks}が、
AzureにはAzure Kubernetes Service(AKS)\cite{aks}が存在するが、
本稿ではAmazon EKSを用いた。

\epsfig{figs/eks.pdf}{width=8.5cm}{Amazon EKS の概要}{eks}

EKSではノードを通常のEC2上の仮想計算機もしくはFargate\cite{fargate}上
に構築する。EKSをもちいることで、非常に大規模な実験環境を
容易に管理運用することができる。



\subsection{MQTT}
MQTT\cite{MQTT}は、Publish/Subscribeモデルに基づく軽量な通信プロトコルであり、広く普及している。
MQTTの通信には、Publisher, Subscriber, Brokerの3者が関与し、Brokerを中心とした
スター構造の通信トポロジをとる\reffig{mqtt}。
Publisherはデータの送信者であり、特定のトピックを指定してBrokerにメッセージを送信する。
Subscriberはデータの受信者であり、Brokerに対して特定のトピックに対する受信希望を行う。
Brokerは、Publisherからのメッセージを受信すると、そのメッセージで指定されているトピック
に対して受診希望を行っているSubscriberに、そのメッセージを送信する。
Brokerを介することで多対多の柔軟な通信が可能となる。

\epsfig{figs/mqtt.pdf}{width=8.0cm}{MQTTの概要}{mqtt}

MQTTには3レベルのQoS(Quality of Service) が用意されている。Publisherは送信時にQoSを指定することができる。
QoS 0はベストエフォートによる送信で、到達性は保証されない。つまりメッセージは途中で破棄される可能性がある。
QoS 1はAt-least-once(少なくとも1回)の到達を保証する。再送を行うため受信通知が失われた場合には複数回
メッセージが配送される可能性がある。
QoS 2はexactly-once(厳密に1回)の到達を保証する。送信側が受信通知に反応するまでライブラリが
メッセージをユーザに引き渡さないことによって、複数回の配送を抑制する。


\section{超分散テストベッド}\label{sec:testbed}
本節では、本稿で構築する超分散テストベッドを概説する。
我々は膨大の数のセンサから得られる情報をクラウドに集積することに興味を持っている。
これをナイーブに実装するとクラウドの負荷が非常に高くなることが予想される。
これに対応するために、センサ群を地理的に局所性を持つグループに分割し、
グループ内で一旦集積および集約してからクラウドに送信する方法が考えられる(\reffig{continuum})。

\epsfig{figs/proposed.pdf}{width=8.5cm}{超分散テストベッドの概要}{continuum}

この中間で集積するサーバを中継局が設置されるエッジに配置し、
デバイス、エッジ、クラウドの3階層で情報を収集する。
エッジで行う集約としては、単純にメッセージを集約してメッセージ数を減らすことや、
平均値や最大最小値としてサマライズするなど、アプリケーションに応じて
さまざまな方法が考えられる。



\section{提案システム}\label{sec:proposal}

\subsection{本稿で構築する環境の概要}
本稿で提案するシステムの目的は、\ref{sec:testbed}節で述べた
環境を擬似的にパブリッククラウド内に構築することである。

具体的には、センサを模したPublisher(以降Sensorと呼ぶ)からの送信を中間層の
Brokerでローカルに集約し、さらにクラウドを模した最終層のBrokerに集約する
2段階の構成を考える
一段目のBrokerをEdgeBroker、2段目のBrokerをCloudBrokerと呼ぶ。
1つのEdgeBrokerが集約するSensorの数を$M$、
EdgeBrokerの数を$N$とする。系全体としては $M \times N$個のSensorが
存在することになる。
EdgeBrokerの背後には、中継と集約を行うポッド(以下Relayと呼ぶ)を配置する。
このRelayはEdgeBrokerに対してはSubscriberとして振る舞い、
CloudBrokerに対してはPublisherとして振る舞う。
つまり、EdgeBrokerからメッセージを受け取り、集約を行った後に
CloudBrokerに送信する。
CloudBrokerに対するSubscriber(Receiverと呼ぶ)が
最終的にメッセージを受け取る。

\subsection{\kbs へのマッピング}

負荷分散のために、各エッジ環境を1つのノードを割り当てる。
またクラウド環境にも1つのノードを割り当てる。
エッジ環境は$N$個あるので、ノード数は $N+1$ となる。

コンテナの種類としては、下記の4種類を用意した。
\begin{itemize}
  \item EdgeBrokerおよびCloudBroker
  \item Sensor 
  \item Relay
  \item Receiber
\end{itemize}

この様子を\reffig{setup}に示す。
エッジ環境のノードには、$M$個のSensorポッドとEdgeBrokerポッド、
およびRelayポッドが動作する。
クラウド環境ノードでは、CloudBrokerポッドとReceiverポッドが動作する。

\epsfig{figs/experimental-setup.pdf}{width=8.5cm}{実験環境の構成}{setup}

すべてのノードでEdgeBrokerもしくはCloudBrokerを動作させるので、
ここではDaemonSetを用いている。これによってノードを起動させる
だけで自動的にBrokerが立ち上がる(EdgeBrokerとCloudBrokerは
同じイメージで作成している)。

また、Sensorの起動にはDeploymentを用いた。
レプリカ数を指定することでノード内に指定した数のSensorを起動することができる。


\subsubsection{証明書の取り扱い}
MQTTはTLS証明書に基づく認証と暗号化をサポートしており、
インターネット環境で使用する際にはこれがデフォルトとなる。
今回の実験ではクライアント認証は行わずサーバ(Broker)側のみで認証を行った。

サーバ認証を行うには、サーバ側にはCA(Certificate Authority)の
証明書とサーバの秘密鍵ファイルを、
クライアント側にはCAの証明書を配置する必要がある。
本来であれば秘密鍵ファイルをサーバから動かすことはできないため、
外部にCAを設置し、サーバでCSRを作成してCAに送付し、
CAで作成した証明書をサーバに送り返す、という手順が必要だが、これは煩雑である。

クラウドに閉じた環境では実際にアクセスされる危険性はなく、
TLSによる負荷の上昇だけを検証することがTLSを使用する目的であるため、
今回は単純化した手続きを採用した。
具体的には、手元のPC上にCAを作成し、そのCAを用いてサーバ証明書と秘密鍵のペアを
1つだけ作り、それらをすべてのポッドで共有した。
ポッドでの共有には前述したSecretを使用した。

\subsection{BrokerのIPアドレスの解決}



\begin{figure}[tbp]
  \renewcommand{\baselinestretch}{0.8}
  \begin{lstlisting}
apiVersion: apps/v1
kind: DaemonSet
metadata:
  name: broker-pod
spec:
  selector:
    matchLabels:
      app: broker
  template:
    metadata:
      labels:
        app: broker
    spec:
      volumes:
      - name: secret-volume
        secret:
          secretName: broker-certs
      containers:
      - name: broker-pod
        image: XXXXX/YYYY:latest
        volumeMounts:
        - name: secret-volume
          mountPath: /etc/secrets/broker-certs
        command:
          - /bin/sh
          - -c
          - >
          中略
          /usr/sbin/mosquitto -c \
          /etc/mosquitto/mosquitto.conf
        ports:
        - containerPort: 1883
          protocol: TCP
        - containerPort: 8883
          protocol: TCP
  \end{lstlisting}
  \caption{EdgeBroker兼CloudBrokerのyamlファイル}
  \label{broker-yaml}
\end{figure}



\subsection{Amazon EKSでの実現}

コンテナイメージの保存にはDocker Hub\cite{dockerhub}を用いる。
実験環境の概要を\reffig{experiment}に示す。

\epsfig{figs/experiment.pdf}{width=8.5cm}{実験の実行}{experiment}


Amazon EKSに対する操作にはAmazonの提供するPython APIであるBoto3\cite{boto3}を
使用した。また、
\kbs のPython用API\cite{kubernetes-python-client}を使用して実装した。


\subsection{実験の手順}
実験の過程は以下のようになる。
\begin{itemize}
  \item 実験を実行するコードを記述する。\\例えばSensorではメッセージを特定のトピックで送出し、Receiverは特定のトピックから受信する。RelayはEdgeBrokerからメッセージを受け取りそれを集積して、CloudBrokerに送出する。
  \item 上記の実験コードを含むコンテナイメージを作成しDocker Hubに登録する。
  \item EKS環境を用意し、必要なノードを起動する。
  \item Jupyter Notebookに書かれた手順にしたがって、EKS上に実験環境をセットアップする。
  \item 実験が完了したらログを取得する。
  \item EKS上の実験環境を破棄する。
  \item EKS環境を破棄する。
\end{itemize}


%\section{評価}\label{sec:evaluation}

\section{おわりに}\label{sec:conclusion}
本稿では、大量のセンサデータを収集する超分散システムの評価を行うテストベッドを
パブリッククラウド上のコンテナオーケストレーションサービスを用いることで
自動化するシステムを提案した。
本システムを用いることで、大規模な環境の実験が比較的容易に行えることを示した。

今後の課題としては、以下が挙げられる。
\begin{itemize}
  \item より大規模な環境の実験\\
  今回の実験では100センサーからなる環境の構築を行った。
  より大規模な環境に対してもスケールすることを確認する。
  
  \item ネットワークのエミュレーション\\
  提案システムのコンテナ間のネットワーク接続は、ネイティブの速度で動作するが、
  より詳細な評価を行うにはネットワークのエミュレーションが必要になる。
  例えば帯域制約やレイテンシのインジェクションなどを検討する。

\end{itemize}

\begin{acknowledgment}
本成果の一部は、国立研究開発法人新エネルギー・産業技術総合開発機構(NEDO)の
「ポスト5G情報通信システム基盤強化研究開発事業」(JPNP20017)の
委託事業の結果得られたものである。
\end{acknowledgment}

\bibliographystyle{ipsjunsrt}
\bibliography{citation}

\end{document}


